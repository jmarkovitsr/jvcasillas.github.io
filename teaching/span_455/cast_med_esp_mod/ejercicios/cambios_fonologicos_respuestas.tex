\documentclass[12pt]{article}
\usepackage{tipa}
\usepackage[english,spanish]{babel}
\usepackage[utf8]{inputenc}
\usepackage{t1enc}

\usepackage[hmargin=2cm,vmargin=2cm]{geometry}
\geometry{a4paper}

\usepackage{fancyhdr} % This should be set AFTER setting up the page geometry
\pagestyle{fancy} % options: empty , plain , fancy
\renewcommand{\headrulewidth}{0pt} % customise the layout...
\lhead{}\chead{}\rhead{\large{Cambios fonológicos}}
\lfoot{}\cfoot{}\rfoot{}


\begin{document}


{\large{
\begin{enumerate}
\itemsep=.5mm
	\item Pérdida de \textipa{[m]} final
	\item Pérdida de \textipa{[h]} inicial
	\item Confluencias vocálicas
	\item Desvelaricación de \textipa{[w]} (> \ \textipa{[B]})
	\item {\textipa{[e]}} en hiato > \ \textipa{[j]}
	\item {\textipa{[t]}} y \textipa{[k]} ante \textipa{[j]} (> \ \textipa{[\textteshlig]} > \ \textipa{[ts]})
	\item {\textipa{[k]}} ante vocales anteriores (> \ \textipa{[\textteshlig]} > \ \textipa{[ts]})
	\item Pérdida de vocales intertónicas (primera fase)
	\item Palatalización de consonantes velares agrupadas
	\item Asimilación de ciertos grupos consonánticos
	\item Palatalización de consonantes ante \textipa{[j]}
	\item Diptongación
	\item Cambio de \textipa{[f]} inicial a \textipa{[h]}
	\item Rehilamiento de \textipa{[\textturny] en [\textyogh]}
	\item Lenición
	\item Palatalización de \textipa{[l]} y \textipa{[n]} geminadas
	\item Palatalización de \textipa{[kl], [pl], [fl]} en posición inicial
	\item Pérdida de vocales intertónicas (segunda fase)
	\item Pérdida de \textipa{[t], [d], [k]} finales
	\item Pérdida de \textipa{[e]} final
	\item Ajuste de grupos consonáticos
	\item Prótesis ante \textipa{[s]} agrupada incial
\end{enumerate}

\hrule

\begin{enumerate}
	\item[23.] Pérdida de \textipa{[h]} inicial, proveniente de \textipa{[f]}
	\item[24.] \textipa{/b/} y \textipa{/\textbeta/} confluyen en \textipa{/b/}  
	\item[25.] Desafricación de \textipa{[ts]} y \textipa{[dz]}  
	\item[26.] Ensordecimiento de las sibilantes sonoras  
	\item[27.] Cambio de articulación de \textipa{[\c{s}]}  
	\item[28.] Cambio de articulación de \textipa{[\textesh]} 
	\item[29.] Yeísmo (\textipa{/\textturny/} > \textipa{/\textctj/})
\end{enumerate}
}}


\clearpage


\noindent \textbf{Aplica, en orden cronológico los 29 cambios fonológicos presentados por Pharies a las palabras siguientes de modo que ilustren la evolución fonológica de las mismas entre el latín común y el español moderno:} \\


\begin{tabular}{lllcl}
	    &                  & {\sc Forma} & {\sc N$^o$ del cambio} & {\sc Explicación} \\ [1ex]
	1.  & aliu `ajo'       & \textipa{[\textprimstress a.li.u]} & & Forma latina\\
	    &                  & \textipa{[\textprimstress a.le.o]} & 3 & Confluencias vocálicas \\ 
	    &                  & \textipa{[\textprimstress a.ljo]}  & 5 & [e] en hiajo > \ [j] \\ 
	    &                  & \textipa{[\textprimstress a.\textturny o]} & 11 & Palatalización de con. ante [j] \\ 
	    &                  & \textipa{[\textprimstress a.\textyogh o]} & 14 & Rehilamiento de [\textturny] en [\textyogh] \\
	    & 				   & \textipa{[\textprimstress a.\textesh o]} & 26 & Ensordecimiento de sibilantes sonoras \\
	    &                  & \textipa{[\textprimstress a.xo]} & 28 & Cambio de articulación de \textipa{[\textesh]} \\ [3ex]
	   

	2.  & anniculu `añejo' & \textipa{[an.\textprimstress ni.ku.lu ]} & & Forma latina \\
	    &                  & \textipa{[an.\textprimstress ne.ko.lo]} & 3 & Confluencias vocálicas \\
	    &                  & \textipa{[an.\textprimstress ne.klo]} & 8 & Pérdida de intertónicas \\
	    &                  & \textipa{[an.\textprimstress ne.\textturny o]} & 9 & Palatalización [kl] > [jl] > [\textturny] \\
	    &                  & \textipa{[an.\textprimstress ne.\textyogh o]} & 14 & Rehilamiento de [\textturny] en [\textyogh] \\
	    &                  & \textipa{[a.\textprimstress \textltailn e.\textyogh o]} & 16 & Palatalización de [n] geminada \\ 
	    &                  & \textipa{[a.\textprimstress \textltailn e.\textesh o]} & 26 & Ensordecimiento de sibilantes sonoras \\
	    &                  & \textipa{[a.\textprimstress \textltailn e.xo]} & 28 & Cambio de articulación de \textipa{[\textesh]} \\ [3ex]


	3.  & apicula `abeja' & \textipa{[a.\textprimstress pi.ku.la]} & & Forma latina\\
	    &                 & \textipa{[a.\textprimstress pe.ko.la]} & 3 & Confluencias vocálicas \\ 
	    &                 & \textipa{[a.\textprimstress pe.kla]} & 8 & Pérdida de intertónicas \\ 
	    &                 & \textipa{[a.\textprimstress pe.\textturny a]} & 9 & Palatalización [kl] > [jl] > [\textturny] \\ 
	    &                 & \textipa{[a.\textprimstress pe.\textyogh a]} & 14 & Rehilamiento de [\textturny] en [\textyogh] \\
	    &                 & \textipa{[a.\textprimstress be.\textyogh a]} & 15 & Lenición \\
	    &                 & \textipa{[a.\textprimstress be.\textesh a]} & 26 & Ensordecimiento de sibilantes sonoras \\
	    &                 & \textipa{[a.\textprimstress be.xa]} & 28 & Cambio de articulación de \textipa{[\textesh]} \\ [3ex]


	4.  & auricula `oreja' & \textipa{[aw.\textprimstress \textfishhookr i.ku.la]} & & Forma latina \\
	    &                  & \textipa{[o.\textprimstress \textfishhookr e.ko.la]} & 3 & Confluencias vocálicas  \\ 
	    &                  & \textipa{[o.\textprimstress \textfishhookr e.kla]} & 8 & Pérdida de intertónicas  \\ 
	    &                  & \textipa{[o.\textprimstress \textfishhookr e.\textturny a]} & 9 & Palatalización [kl] > [jl] > [\textturny] \\
	    &                  & \textipa{[o.\textprimstress \textfishhookr e.\textyogh a]} & 14 & Rehilamiento \\
	    &                  & \textipa{[o.\textprimstress \textfishhookr e.\textesh a]} & 26 & Ensordecimiento de sibilantes sonoras \\
	    &                  & \textipa{[o.\textprimstress \textfishhookr e.xa]} & 28 & Cambio de articulación de \textipa{[\textesh]} \\ [3ex]


	5. & clausa `llosa'   & \textipa{[\textprimstress klaw.sa]} & & Forma latina \\
	    &                  & \textipa{[\textprimstress klo.sa]} & 3 & Confluencias vocálicas \\
	    &                  & \textipa{[\textprimstress klo.za]} & 15 & Lenición \\
	    &                  & \textipa{[\textprimstress \textturny o.sa]} & 17 & Palatalización [kl] > [jl] > [\textturny] \\
	    &                  & \textipa{[\textprimstress jo.sa]} & 29 & Yeísmo \\ [3ex]


	6. & cond\={u}x\={\i} `conduje' & \textipa{[kon.\textprimstress du:k.si:]} & & Forma latina \\
	    &                     & \textipa{[kon.\textprimstress duk.si]} & 3 & Confluencias vocálicas \\
	    &                     & \textipa{[kon.\textprimstress du.Se]} & 9 & Palatalización [ks] > [js] > [\textipa{S}] \\ 
	    &                     & \textipa{[kon.\textprimstress du.xe]} & 28 & Cambio de articulación de \textipa{[\textesh]} \\ [3ex]

\end{tabular}

\clearpage

\begin{tabular}{lllcl}
	    &                  & {\sc Forma} & {\sc N$^o$ del cambio} & {\sc Explicación} \\ [1ex]
	7. & corticia `corteza'  & \textipa{[ko\textfishhookr.\textprimstress ti.ki.a]} & & Forma latina \\
	    &                     & \textipa{[ko\textfishhookr.\textprimstress te.ke.a]} & 3 & Confluencias vocálicas \\
	    &                     & \textipa{[ko\textfishhookr.\textprimstress te.kja]} & 5 & [e] en hiato > [j] \\
	    &                     & \textipa{[ko\textfishhookr.\textprimstress te.tsa]} & 6 & [k] ante [j] > [\textteshlig] > [ts] \\
	    &                     & \textipa{[ko\textfishhookr.\textprimstress te.dza]} & 15 & Lenición \\
	    &                     & \textipa{[ko\textfishhookr.\textprimstress te.\c{z}a]} & 25 & Desafricación de \textipa{[dz]} \\
	    &                     & \textipa{[ko\textfishhookr.\textprimstress te.\c{s}a]} & 26 & Ensordecimiento de sibilantes sonoras \\
	    &                     & \textipa{[ko\textfishhookr.\textprimstress te.\texttheta a]} & 27 & Cambio de articulación de \textipa{[\c{s}]} \\ [3ex]

    8. & decimu `diezmo'     & \textipa{[\textprimstress de.ki.mu]} & & Forma latina \\
        &                     & \textipa{[\textprimstress dE.ke.mo]} & 3 & Confluencias vocálicas \\
        &                     & \textipa{[\textprimstress dE.tse.mo]} & 7 & [k] ante voc. anterior > [\textteshlig] > [ts] \\
        &                     & \textipa{[\textprimstress dEts.mo]} & 8 & Pérdida de intertónicas \\
        &                     & \textipa{[\textprimstress djets.mo]} & 12 & Diptongación \\ [3ex]

	9.	& d\={\i}cit `dice' & \textipa{[\textprimstress di:.kit]} & & Forma latina \\ 
        & 					& \textipa{[\textprimstress di.ket]} & 3 & Confluencias vocálicas \\
        & 					& \textipa{[\textprimstress di.tset]} & 7 & \textipa{[k]} ante voc. anterior > \textipa{[\textteshlig]} > \textipa{[ts]} \\
        & 					& \textipa{[\textprimstress di.dzet]} & 15 & Lenición \\
        & 					& \textipa{[\textprimstress di.dze]} & 19 & Pérdida de \textipa{[t]} final \\
        & 					& \textipa{[\textprimstress di.\c{z}e]} & 25 & Desafricación de \textipa{[dz]} \\
        & 					& \textipa{[\textprimstress di.\c{s}e]} & 26 & Ensordecimiento de sibilantes sonoras \\
        & 					& \textipa{[\textprimstress di.\texttheta e]} & 27 & Cambio de articulación de \textipa{[\c{s}]} \\ [3ex]

	10.	& facit `hace' & \textipa{[\textprimstress fa.kit]} & & Forma latina \\ 
	    &              & \textipa{[\textprimstress fa.ket]} & 3 & Confluencias vocálicas \\ 
	    &              & \textipa{[\textprimstress fa.tset]} & 7 & \textipa{[k]} ante voc. anterior > \textipa{[\textteshlig]} > \textipa{[ts]} \\ 
	    &              & \textipa{[\textprimstress ha.tset]} & 13 & Cambio de \textipa{[f]} inicial a \textipa{[h]} \\ 
	    &              & \textipa{[\textprimstress ha.dzet]} & 15 & Lenición \\ 
	    &              & \textipa{[\textprimstress ha.dze]} & 19 & Pérdida de \textipa{[t]} final \\ 
	    &              & \textipa{[\textprimstress a.dze]} & 23 & Pérdida de \textipa{[h]} inicial \\
	    &              & \textipa{[\textprimstress a.\c{z}e]} & 25 & Desafricación de \textipa{[dz]} \\
	    &              & \textipa{[\textprimstress a.\c{s}e]} & 26 & Ensordecimiento de sibilantes sonoras \\
	    &              & \textipa{[\textprimstress a.\texttheta e]} & 26 & Cambio de articulación de \textipa{[\c{s}]} \\ [3ex]

	11.	& lumbr\={\i}ce `lombriz' & \textipa{[lum.\textprimstress b\textfishhookr i:.ke]} & & Forma latina \\ 
		&                         & \textipa{[lom.\textprimstress b\textfishhookr i.ke]} & 3 & Confluencias vocálicas \\ 
		&                         & \textipa{[lom.\textprimstress b\textfishhookr i.tse]} & 7 & \textipa{[k]} ante voc. anterior > \textipa{[\textteshlig]} > \textipa{[ts]} \\ 
		&                         & \textipa{[lom.\textprimstress b\textfishhookr i.dze]} & 15 & Lenición \\ 
		&                         & \textipa{[lom.\textprimstress b\textfishhookr idz]} & 20 & Pérdida de [e] final \\ 
		&                         & \textipa{[lom.\textprimstress b\textfishhookr i\c{z}]} & 25 & Desafricación de \textipa{[dz]} \\ 
		&                         & \textipa{[lom.\textprimstress b\textfishhookr i\c{s}]} & 26 & Ensordecimiento de sibilantes sonoras \\ 
		&                         & \textipa{[lom.\textprimstress b\textfishhookr i\texttheta]} & 27 & Cambio de articulación de \textipa{[\c{s}]} \\ [3ex]

\end{tabular}

\clearpage

\begin{tabular}{lllcl}
	    &                  & {\sc Forma} & {\sc N$^o$ del cambio} & {\sc Explicación} \\ [1ex]
	12.	& malitia `maleza' & \textipa{[ma.\textprimstress li.ti.a]} & & Forma latina \\ 
		&                  & \textipa{[ma.\textprimstress le.te.a]} & 3 & Confluencias vocálicas \\ 
		&                  & \textipa{[ma.\textprimstress le.tja]} & 5 & \textipa{[e]} en hiato > \textipa{[j]} \\ 
		&                  & \textipa{[ma.\textprimstress le.tsa]} & 6 & \textipa{[t] ante [j]} > \textipa{[\textteshlig]} > \textipa{[ts]} \\ 
		&                  & \textipa{[ma.\textprimstress le.dza]} & 15 & Lenición \\ 
		&                  & \textipa{[ma.\textprimstress le.\c{z}a]} & 25 & Desafricación de \textipa{[dz]} \\ 
		&                  & \textipa{[ma.\textprimstress le.\c{s}a]} & 26 & Ensordecimiento de sibilantes sonoras \\ 
		&                  & \textipa{[ma.\textprimstress le.\texttheta a]} & 27 & Cambio de articulación de \textipa{[\c{s}]} \\ [3ex]

	13.	& martiu `marzo' & \textipa{[\textprimstress ma\textfishhookr.ti.u]} & & Forma latina \\ 
		&                & \textipa{[\textprimstress ma\textfishhookr.te.o]} & 3 & Confluencias vocálicas \\ 
		&                & \textipa{[\textprimstress ma\textfishhookr.tjo]} & 5 & \textipa{[e]} en hiato > \textipa{[j]} \\ 
		&                & \textipa{[\textprimstress ma\textfishhookr.tso]} & 6 & \textipa{[t] ante [j]} > \textipa{[\textteshlig]} > \textipa{[ts]} \\ 
		&                & \textipa{[\textprimstress ma\textfishhookr.dzo]} & 15 & Lenición \\ 
		&                & \textipa{[\textprimstress ma\textfishhookr.\c{z}o]} & 25 & Desafricación de \textipa{[dz]} \\
		&                & \textipa{[\textprimstress ma\textfishhookr.\c{s}o]} & 26 & Ensordecimiento de sibilantes sonoras \\
		&                & \textipa{[\textprimstress ma\textfishhookr.\texttheta o]} & 27 & Cambio de articulación de \textipa{[\c{s}]} \\ [3ex]
	

	14.	& meli\={o}re `mejor' & \textipa{[me.li.\textprimstress o:.\textfishhookr e]} & & Forma latina \\ 
		&                     & \textipa{[me.le.\textprimstress o.\textfishhookr e]} & 3 & Confluencias vocálicas \\ 
		&                     & \textipa{[me.\textprimstress ljo.\textfishhookr e]} & 5 & \textipa{[e]} en hiato > \textipa{[j]} \\ 
		&                     & \textipa{[me.\textprimstress \textturny o.\textfishhookr e]} & 11 & Palatalización de con. ante [j] \\ 
		&                     & \textipa{[me.\textprimstress \textyogh o.\textfishhookr e]} & 14 & Rehilamiento \\ 
		&                     & \textipa{[me.\textprimstress \textyogh o\textfishhookr]} & 20 & Pérdida de [e] final \\
		&                     & \textipa{[me.\textprimstress \textesh o\textfishhookr]} & 26 & Ensordecimiento de sibilantes sonoras \\
		&                     & \textipa{[me.\textprimstress xo\textfishhookr]} & 28 & Cambio de articulación de \textipa{[\textesh]} \\ [3ex]

	15.	& paus\={a}re `posar' & \textipa{[paw.\textprimstress sa:.\textfishhookr e]} & & Forma latina \\
		&                     & \textipa{[po.\textprimstress sa.\textfishhookr e]} & 3 & Confluencias vocálicas \\
		&                     & \textipa{[po.\textprimstress sa\textfishhookr]} & 20 & Pérdida de [e] final \\


\end{tabular}


\end{document}